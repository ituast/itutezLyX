Rubber bearings are used to prevent vibrations in the buildings and
to allow the bearings to be displaced due to thermal expansion in
the bridges. The first use of rubber bearings in order to protect
constructions against earthquake effects, occurred in Pestalozzi primary
school in Skopje, Yugoslavia in 1969. The same horizontal and vertical
stiffness of the rubber supports applied as a single block caused
the bulge to occur due to the weight of the building on the side surfaces.
The French engineer Eugène Freyssinet, who discovered that the axial
loading capacities of the rubber layers were inversely proportional
to their height, suggested strengthening the rubber layers by adding
thin steel plates in the vertical direction. Here the bond between
the layers is provided due to the friction force. Thanks to the vulcanization
method used to ensure that the thin steel plates and rubber layers
adhere to each other, studies and applications of modern seismic isolators
have begun to increase.

Seismic isolation systems are frequently used to prevent damage to
structures in areas with high seismicity due to strong ground motions.
Base isolated structures consist of superstructure, isolation plane
and base isolators. These systems are single structures such as residential,
data center building, hospital, liquid tank and offshore oil platform
which are assumed to displace relatively rigid over the base isolators.
The architectural details to be solved in the case of using separate
base isolated structures together, are complex and expensive. For
this reason independent superstructures are designed in common isolation
plane. More particularly, constructed hospital buildings in recent
years in Turkey, meets this definition.

Linear design methods, which are determined in consequence of carried
out studies in the past, are consider the base isolated systems as
consist of single or multi degree of freedom systems. Therefore design
procedures in standards are prepared for base isolated structures
with independent isolation plane. There is no design criteria in standards
for dynamic interaction of base isolated structures with common isolation
plane. Each of the base isolated structures which are designed in
common isolation plane are considered in practice as if structures
with independent isolation plane. As a consequence of this assumption
causes the higher mode effect due to the hysteretic behavior of the
isolators and therefore the dynamic interaction of superstructures
to be unevaluated.

Purpose of this study is to determine amplification of the base shear
coefficient due to interaction of superstructures by performing comprehensive
parametric investigation for the two base isolated structures with
common isolation plane.

Within the scope of this study, dynamic interaction of two base isolated
structures with common isolation plane is investigated through the
systems changes parametrically. In addition to this sample analyses
are carried out for three and four base isolated structures with common
isolation plane. In this case, number of story of superstructures,
equivalent period and equivalent damping ratio of base isolators are
considered as parameter. Analyses are carried out for the cases where
the number of story of two base isolated structures with common isolation
plane changes from one to ten. All analyses are repeated for 1.5,
2.5, 4.0s equivalent period and \%10, \%20, \%30 equivalent damping
ratio of base isolators. The hysteretic properties of the base isolators
are normalized to obtain constant equivalent period and equivalent
damping ratios as a result of nonlinear analysis versus varying superstructure
mass. This process is performed by determining the yielded stiffness
and characteristic strength by determining the isolator displacements
obtained from the linear time-history analysis with the stiffness
corresponding intended equivalent period and the equivalent damping
values.

Analyses are carried out only for the direction in which the structures
are located in the common insulation plane. The effects that occur
other directions and the bi-directional interactions of the isolators
are out of scope of this study. Story mass and story stiffnesses of
superstructures are fixed for all analyses. Mass of the isolation
plane is fixed for each structure and is calculated by multiplying
the number of structures in the common isolation plane. Only linear
elastic shear springs are considered in superstructure models which
are defined as multi degree of freedom system. Hysteretic behavior
of isolators are represented by a bilinear element. Only hysteretic
damping which is caused by nonlinear behavior is considered in base
isolators. The ground motion records used in the analyses are matched
in accordance with the design spectrum generated according to earthquake
ground motion level with probability of exceeding \%2 in 50 years.

All analyzes were carried out by means of the MSBIS program, which
was developed within the scope of this study, which allows nonlinear
analysis of base isolated structures with independent and common isolation
plane to be solved parametrically. Numerical solution of equation
of motions is carried out with Newmark-$\beta$ method. Dynamical
balance is provided by Newton-Raphson method doing iteration in every
time step. The accuracy of the developed program is demonstrated by
comparison with SAP2000, the generally accepted structural analysis
program.

Resultant shear forces, story accelerations and story drifts of the
first structure is determined by means of nonlinear time history analyses
according to changing angular frequency of the second structure and
effective period and damping values of base isolators. In addition,
for the case of first structure have ten stories, the variation of
story shear force coefficients for the changing number of story of
the second structure is examined. Results are normalized according
to values which obtain from the inspected base isolated structures
with independent isolation plane. Thus, amplification in base shear
forces of the structures can be determined depending on the angular
frequency of the superstructures, equivalent period and equivalent
damping of base isolator. Base shear force coefficients of the structures
with common isolation plane are obtained and the results are compared
according to the total shear force coefficient obtained by vector
summation in the dynamic analysis. This rate of error is increasing
in direct proportion to the order of the dynamic interaction of the
structures. Additionally necessary joint spacing of two base isolated
structures with common isolation plane is calculated as per related
standard and compared to results obtained from the nonlinear dynamic
analysis.

As a result of this study, it is determined that dynamic interaction
of two base isolated structures with common isolation plane due to
increasing damping ratio of base isolators, causes to increase shear
coefficients, story drifts and maximum story accelerations significantly.
This increment is determined to increase with the separation of the
angular frequencies of the two structures. Therefore, if two structures
have the same angular frequency, the resulting internal forces and
displacements are equal for the same equivalent period and equivalent
damping ratios base isolators. Base shear force of one story structure
inspected from two base isolated structures with common isolation
plane, increase with respect to vary other structure number of story
according to situation that inspected structure with independent isolation
plane. When the ten-story structure is examined, it is determined
that the story shear forces increase in the lower stories and decrease
in the upper stories. As a result of the analyses carried out for
the three and four base isolated structure with common isolation plane,
it is determined that the base shear force coefficients increase with
the increase of the number of buildings having different angular frequency
from the examined structure. Moreover, increasing in hysteretic damping
causes to change lateral distribution of base shear forces to superstructure.
Base shear force of structure is evenly distributed to every story
for an equivalent damping value of \%1, which is assumed to be linear
of the isolators. Nevertheless, base shear force of structure is distributed
in the form of an inverted triangle for increasing damping ratios.
